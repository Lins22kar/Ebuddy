Testing is the process of evaluating a system or its component(s) with the intent to find
whether it satisfies the specified requirements or not. In simple words, testing is executing a
system in order to identify any gaps, errors, or missing requirements in contrary to the actual
requirements. Software Testing is a process of verifying and validating whether the program
is performing correctly with no bugs. It is the process of analyzing or operating software
for the purpose of finding bugs. It also helps to identify the defects / aws / errors that
may appear in the application code, which needs to be fixed. Testing not only means fixing
the bug in the code, but also to check whether the program is behaving according to the
given specifications and testing strategies. This Chapter is organized as follows.

Section 6.1
describes Black Box Testing and white Box Testing. Section 6.2 describes Manual Testing
and Automated Testing. Test Cases Identification and Execution describe in Section 6.3.
Finally, summary of the chapter is given in last section

\section{Black Box and White Box Testing}
The following methodologies are used for testing.
\subsection{Black Box Testing}
Black Box testing also known as Behavioral Testing, is a software testing method in which
the internal structure/design/implementation of the item being tested is not known to the
tester. These tests can be functional or non-functional, though usually functional.
This method is named so because the software program, in the eyes of the tester, is like a
black box; inside which one cannot see. This method attempts to find errors in the following categories
\begin{itemize}
\item Incorrect or missing functions
\item Interface errors
\item Errors in data structures or external database access
\item Behaviour or performance errors
\item Initialization and termination errors
\end{itemize}

\subsection{White Box Testing}
White Box Testing is also known as Clear Box Testing, Open Box Testing, Glass Box Testing,
Transparent Box Testing, Code-Based Testing or Structural Testing. It is a software testing
method in which the internal structure, design, implementation of the item being tested is
known to the tester. The tester chooses inputs to exercise paths through the code and determines the appropriate outputs. Programming know-how and implementation knowledge
is essential. White box testing is testing beyond the user interface and into the nitty-gritty
of a system.
This method is named so because the software program, in the eyes of the tester, is like
a white or transparent box; inside which one clearly sees.

\section {Manual and Automated Testing}
Following methodologies are used for testing.
\subsection{Manual Testing}
It is the oldest and most rigorous types of testing it is performed by human sitting in
front of a computer carefully going through applications screens, trying various usage and
input combinations, comparing the results to the expected behavior and recording their
observations about project. There are certain ways of manual testing first of all test cases
are written then they are executed and then report is generated according to test result.
\subsection{Automated Testing}
Automation testing is a Software testing technique to test and compare the actual outcome
with the expected outcome. This can be achieved by writing test scripts or using any
automation testing tool. Test automation is used to automate repetitive tasks and other
testing tasks which are difficult to perform manually.
The benefit of manual testing is that it allows a human mind to draw insights from a
test that might otherwise be missed by an automated testing program. Automated testing
is well-suited for large projects; projects that require testing the same areas over and over;
and projects that have already been through an initial manual testing process.

\section{Test Case Identification and Execution}
\subsection{Test Cases}
Test Case is the set of inputs along with the expected output and actual output some
additional information.


\begin{center}
     \begin{tabular}{c|c|c|c|c|c|}
         
         \begin{tabularx}{1\textwidth} { 
  | >{\raggedright\arraybackslash}X 
  | >{\raggedright\arraybackslash}X
  | >{\raggedright\arraybackslash}X 
  | >{\raggedright\arraybackslash}X
  | >{\raggedright\arraybackslash}X
  | >{\raggedright\arraybackslash}X|}
 \hline
  ID & Scenario & Input & Expected Output & Actual output & Result\\
 \hline
 1&Information without any fabrication &Image/text&No fabrication or false information detected&No fabrication or false information detected& Pass\\
\hline
  2&Information with fabrication  &Image/text&Fabrication detected
&Fabrication detected&Pass\\
\hline
 3&Information with fabrication & Image/text & Fabrication detected &Fabrication not detected&Fail \\
\hline
  4&Valid information gathered& Image/text & Information is directed to authorities &Information is directed to authorities &Pass \\
\hline
 5&Valid information gathered & Image/text & Information is directed to authorities&Information is not directed to authorities & Fail
 \\
\hline
\end{tabularx} 
     \end{tabular}
     \caption{Test Cases }
    \label{tab:my_label}
 \end{center}


 
    

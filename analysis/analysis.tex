
The development of computer-based information system includes the system analysis phase which produces or enhances the data model which itself is to creating or enhancing a database.   There are a number of different approaches to system analysis.    The analysis is the process which is used to analyze, refine and scrutinize the gathered information of entities in order to make consistence and unambiguous information. Analysis activity provides a graphical view of the entire System. System Analysis is the process of gathering and interpreting facts, diagnosing problems and using the facts to improve the system. System analysis chapter will show overall system analysis of the concept, description of the system, meaning of the system. System analysis is the study of sets of interacting entities, including computer system analysis.

The organization of this Chapter is as follows. Section 3.1 represents Requirement Collection and Identification. Software Requirement and Specification are described in the Section
3.2. Section 3.3 describes summary of the chapter.

\section{Requirement Collection and Identification}
Requirement collection is the process which is used to gather, analyze, and documentation and reviews the requirements. Requirements describe what the system will do in place of how. In practical application, most projects will involve some combination of these various methods in order to collect a full set of useful requirements. Requirements collection is initiated when the project need is first identified and the project “solution” is to be proposed. Requirements refinement continues after the project is “selected” and as the scope is defined, aligned and approved.

The system will require only images of the road to be analysed which will be uploaded by user either through the web-application or CLI.

\section{Software Requirements Specifications (SRS)}
Software Specification will provide a broad understanding of the requirement specification of this system. Also, understand features of this system along with the requirements. Software Requirement Specification documents guide the developers in the development process and it will help to reduce the ambiguity of the requirements provided by the end-user. It’s used to provide critical information to multiple teams — development, quality assurance, operations, and maintenance. This keeps everyone on the same page.
\subsection{Product Feature}
The product features are high level attributes of a software or product such as software performance, user-friendly interface, security portability, etc. These attributes are defined according to the product, in this case, a software product.\\
They are as follows:
\begin{itemize}
\item The user will be able to upload the name, description and image of the child labour.
\item The user will be able to view the results of submitted images.
\item Our websites enables retailers to set up online shops, customers to browse through the shops, and system administrator to approve and reject requests for new shops and maintain lists of shop categories.
\item The user will be provided with gift cards for their contribution towards the welfare of children.
\end{itemize}

\subsection{Operating Environment}
The software will operate within the following environment:
\begin{itemize}
    \item  Operating System: Windows 7 or later/Linux/MacOS
    \item Java environment required.
    \item Any system with at least 2GB RAM
    \item System with processor Intel CORE i3 or later
\end{itemize}

\subsection{Assumption}

\begin{itemize}
\item It is assumed that the web portal will load and render correctly and as expected on the operating machine.
\item It is assumed that the user will have a working internet connection with sufficient internet speed.
\item It is assumed that the user is able to either upload images through web interface or CLI.
\item It is assumed that the user will upload the images within the given specifications.
\end{itemize}


\subsection{Functional Requirements}
Functional requirements are the functions which are expected from the software or platform. Functional requirements along with requirement analysis help identify missing requirements. They help clearly define the expected system service and behavior.\\ Functional requirements are as follows:
\begin{itemize}
    \item To be able to upload the images through the web interface and command line interface.
    \item To be able to view the results of analysis through web as well as command line interface.
\end{itemize}

\subsection{Non-Functional Requirements}
Non-functional Requirement is mostly quality requirement. That stipulates how well the portal does, what it has to do. Other than functional requirements in practice, this would entail detail analysis of issues such as availability, security, usability and maintainability.\\
Non-functional requirements are as follows:
\begin{itemize}
    \item The processing system is fast enough to reduce the existing delays.
    \item The interface should be simple and minimal.
    \item The results should be comprehensive and detailed. 
\end{itemize} 

\subsection{External Interfaces}

\begin{itemize}
 \item[$\blacksquare$] {\large User interface}\\
The proposed system has several options for users to interact with. Following are the user interfaces available:
 \item Web-application (GUI)
 \item Command Line Interface (Terminal based interface)\\
The web application will be available so that the users will be able to upload images through a simple GUI and the CLI will be available so that the user will be able to integrate it with other systems easily.

\item[$\blacksquare$] {\large Software interface}\\
The only software interface required for this project is the Application Programming Interface with the Jupyter Notebook which will then process the images. This software interface will run on local server along with the Jupyter server.
\end{itemize}

\section{Summary}
In the chapter, Analysis was presented which included the hardware and software require- ments, functional and non-functional requirements and the software requirements specifi- cation(SRS) as well. In the next chapter, Design is described along with various UML diagrams.